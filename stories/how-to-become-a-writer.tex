\documentclass[11pt,letterpaper]{article}
\usepackage{setspace, fullpage, pslatex}

\title{How to become a writer}
\author{Shrutarshi Basu}
\date{\today}

\begin{document}
\maketitle
\doublespacing

In order to learn how to become a writer first learn how to frickin' TYPE. And I don't mean how to hunt and peck and finally find the key you were looking for five minutes after you started looking for it. I mean learn to really type, so that you can spit out a hundred words without breaking a sweat (or your fingers for that matter). Your brain comes up with ideas at the speed of lightning (literally) and if your fingers can't keep up then you're going to end up missing most of those ideas (some of which might be your best). So learn to type. Please. Now. And when you're done come back and read the rest of this.

Done? Good. Now that you know how to type, it's time to figure out \textit{what} to type. As hard as it is to learn how to type fast, it's not going to come to much if what you type is complete garbage. Not to discourage you or anything, I'm just saying that what you'll start off writing is going to be pretty darn bad. But that's ok, you already knew that and that's why you're reading this, right?

Anyway, getting back to how to become a writer. Well, truth be told, I haven't quite figured out that part yet. Don't worry I have the rest of it all figured out. The whole publish, sell, profit, retire to Silicon Valley (don't ask), I have that all down. It's the writing part that's turning out to be a bitch. I'm making progress of course, which is self-evident from the fact that I am actually writing this. So here's what I've got so far. 

Idea number 1: Read. A lot. And then read some more. Most importantly, don't just read what you want to write. By all means read most of what you eventually see yourself writing, study it in fact, but don't limit yourself to that. There's a reason that English majors study all sorts of literature even if they eventually want to become just playwrights or novelists. The more ideas you have swimming in your head, the better you will be at coming up with interesting ones of your own.

Talking about English majors, that's tip number 2: don't be one. It's my theory that English majors make the worst writers. Just think about it: Mark Twain was a steamboat pilot-in-training. Isaac Asimov was a professional chemist. As for Shakespeare they didn't even have English majors in those days, so he certainly doesn't count. Try an interesting major, like engineering, chemistry or applied mathematics. If you must be an English major, make sure it's for some higher purpose (such as there's a really cute girl in your Creative Writing class, but you're simply too shy to actually talk to her). And even then, at least attach yourself to some sort of self-destructive socially-frowned-upon bad habit. I've found alcohol, drugs, schizophrenia and recurring suicide attempts to be the most productive. In the spirit of true science, I am in the process of testing this theory and will let you know the result when I have a few bestsellers under my belt.

At some point in your writing career you will decide that it's important that you read up on the craft of writing. You will scour the library for books on writing and will come upon such titles as Strunk and White's ``The Elements of Style'' and Stephen King's ``On Writing''. Now reading about writing is fine, I've done it myself. But in many ways it will soon begin to resemble the age long art of masturbation. It's along the same lines as reading healthy living books and ordering cheese fries for lunch or reading blogs on entreneurship while keeping your boss' demands at the top of your to-do list. It gives a great sense of doing something worthwhile and gives you absolutely nothing to show for it at the end of the day. 

The remedy to this unfortunate malady of the soul is Tip 3. Are you ready for it? Sure? Okay here goes, take a deep breath. What you need to do to really be a writer is to *write* (or in our case, type). I know it sounds hard and seems like the last thing you would say to someone who wants to be a writer, but believe me, it's true. No, 140 character Twitter updates don't count. Nor does the love-letter you wrote to your third grade English teacher (which by the way, had such terrible spelling and grammar that your teacher would almost certainly have not gone out with you).

What's that you say? What is this \textit{writing} thing I speak of? Funny you should ask. The reason that I talked about reading books on writing is that I've actually read some of them myself. In "On Writing" Stephen King says that writing is basically telepathy: you take your thoughts and put them into other people's heads. That's all well and dandy, but there are two problems: one is that it's pretty hard to put what you're thinking into a form where other people will actually read it and think the same thing that you were. The more important problem is that by and large, most people's thoughts are complete and utter garbage and should be treated as contagious diseases. *Writing* is actually two very separate things: it's communicating what you want other people to think and its thinking thoughts that are worthy to be put into other people's heads.

Tying it all together, let's reflect on the morals of this story for a moment. The first moral is that this really isn't a story, because I have no clue as to how to actually write a story. Second: I'm really not the person to ask if you want to know about how to be a writer. Third: after I write a few awesome bestsellers I will call up all my English major acquaintances from college and laugh derisively in their faces.

The End.
\footnotetext{
  This work is licensed under the Creative Commons Attribution-Noncommercial-Share Alike 3.0 Unported License. To view a copy of this license, visit \texttt{http://creativecommons.org/licenses/by-nc-sa/3.0/} or send a letter to Creative Commons, 171 Second Street, Suite 300, San Francisco, California, 94105, USA.}
\end{document}