\documentclass[11pt,letterpaper]{article}
\usepackage{setspace, fullpage, pslatex}

\title{You must not come lightly}
\author{Shrutarshi Basu}
\date{\today}

\begin{document}
\maketitle
\doublespacing

As James looked at the bright computer screen, he barely saw anything in front of him. The only thing he was aware of was a voice in the back of his head saying over and over  ``You must not come lightly to the blank page''. It was his own voice, but the words were not his own. It was something he remembered from his college days. He couldn't remember where exactly he read it and he thought it ironic that of all the things he learned in the college this was the one thing that came immediately to mind.

With the voice still in his head, he looked from his fingers on the keyboard to the computer screen and back to the keyboard. He thought his fingers should be moving, and was concerned that they weren't. He willed them to move, but they wouldn't. The blank page troubled him. That and the fact that in a little over eight hours it had to be completely full. Though James had filled lots of pages in his lifetime, filling this one turned out to be a surprisingly difficult endeavor. And the voice in his head was starting to get irritating. With nothing on the screen and no movement on the keyboard, his mind drifted to the voice. Why was it there? How did it get there? And why did it keep telling him that he must not come lightly to the blank page? And then it struck him, that for the first time in his life, he actually had come lightly to the blank page.

James' had been in a relationship with the blank page for a long time. So long ago in fact, that he barely remembered when it had started. It had started as such things usually do, with a troubled childhood when the blank page had been an escape from the messiness of the world around him. He had not come lightly to the page, and a lot of times he had come to it only because it was the only place he had left to go to.

He remembered his first journal which was really nothing but a spiral bound notebook. He remembered writing with all sorts of strange things - crayons, pencils, a toothpick dipped in chocolate sauce. He had probably loved it because he cried for days when he thought he had lost it. Years later, he laughed for days when he found it while packing for college.

Throughout his childhood and teenage years, the blank page had transformed from a last refuge to a favorite haunt. His journal had become a continuous record of his life. Its blank pages recorded the first time he met Susan and the last time. It recorded all the days when he lay in bed with a shattered thigh bone because of a series of bad decisions one icy winter night. The pages were filled first by pencil and then by pen. The content spanned all manner of things, both mundane and profound.

James couldn't quite remember when he had decided to make filling the blank page his life's work. It was certainly before college and certainly after he met Susan. But it was still quite a while ago. The blank pages of the diary gave way to the empty screen of a word processor and the pen was replaced by an keyboard. He still had a diary, but he found himself using it less and less. But as he looked at his immobile fingers he thought that it might be helpful to pick up a pen and get some paper.

At some point in college it had struck him that the whole business of writing was becoming easier. He still didn't come lightly to the page, but it was certainly getting easier for him to fill the blanks. Was it when we has twenty? About ten years after he started regularly writing in his diary? That sounded right, but he couldn't be sure. Time seemed to be getting blurry now. He could only think about the eight hours before him, now seven.

He managed to wrest his eyes away from his hands and the screen to look at the room around him, hoping for some inspiration. The apartment was small, but to his taste and the neighbors were quiet, unlike college. Outside he could see the streetlights and make out some of the houses. He half expected it to be raining, a fitting companion to his despair, but it wasn't. There were no raindrops, but there was no inspiration either.

Perhaps that was what he should fill the current blank page with: what did it actually mean to not come lightly to a blank page? Did it mean that he shouldn't take his writing for granted? Did it mean that he shouldn't start writing without knowing what he was writing about? Did it mean that he shouldn't fill a page just for the sake of filling it? Strangely enough, James realized that he didn't quite now. Ever since he had read those words, he thought that he had understood them perfectly. But now that he tried to put his understanding on the screen, the words he needed wouldn't come. Did it mean that he couldn't sit down at a blank page and always expect the words to flow?

At this point in his idle reverie, with six and a half hours to go, he felt a sudden wave of fear. He had never quite understood the concept of a writer's block. As long as he could remember, every time he sat down with the intent to write, he had written. Not that it was all good, much of it had never been seen by another person, but he had still filled the page. Every time, except now. James had never understood when his classmates and even his seniors complained of writer's block. He saw it as a flaw in them that they could not write whenever they chose to. They came lightly to the blank page, expecting to get something for nothing. He had never done that, except now.

Six hours. The clock kept ticking and his fingers refused to move. How long had they been still? One more question to which he didn't know the answer. The nascent fear in his brain was starting to turn to full blown panic. His first paid assignment out of college and here he was, with nothing to show at his first deadline. If the situation had not been quite urgent, he would have enjoyed examining his own emotions, taking them apart and putting them down in words. But under the circumstances, all his powers of description learned over the years seemed to have evaporated.

Summoning his willpower he made an attempt to write, something. His right index finger moved left from the letter `J' on which it had been resting to the letter `H', pressed and released. The left middle finger moved up from `D' to `E' in a similar motion. Right after that his right hand moved to place the ring finger on the backspace . 2 taps later he was back to the blank page.

He had to write. But what? For the last two hours all he had thought of was about how he had written before. And that he must not come lightly to the blank page. It seemed a simple enough condition. It was only then that he realized that if he had spent the last two hours thinking about writing, it wouldn't take him more than three to put it all down. He had spent years learning about writing, practicing his craft, but he had never actually written about his writing. He had come lightly to the blank page, and it had taken him two hours to get rid of that lightness. Now he had the weight of purpose with him and that weight would drive his fingers. In the time that it had taken for him to think these thoughts, the page had ceased to be blank. It contained the words : ``I would like to write about writing''.
\footnotetext{
  This work is licensed under the Creative Commons Attribution-Noncommercial-Share Alike 3.0 Unported License. To view a copy of this license, visit \texttt{http://creativecommons.org/licenses/by-nc-sa/3.0/} or send a letter to Creative Commons, 171 Second Street, Suite 300, San Francisco, California, 94105, USA.}
\end{document}