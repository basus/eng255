\documentclass[12pt,letterpaper]{article}
\usepackage{setspace, fullpage, pslatex}

\title{Some technical notes}
\author{Shrutarshi Basu}
\date{\today}

\begin{document}

\maketitle
\doublespacing

For some years now I have had an interest in finding efficient ways to use technology to aid the creation and distribution of written content. In particular, I've developed a disdain for modern word processors: they clutter the creative process by requiring the writer to keep in mind such inanities as font size and style, margin size and other such details which computers should be able to handle automatically.

I took the opportunity to use this class as a testbed for some ideas that I had hoped would help my creative process. My first decision was to stop using word processing and instead writing in plain text files. For printing purposes I used a system called \LaTeX. It is designed for state-of-the-art typesetting and produces gorgeous print documents. This allowed to me focus on \textit{writing} leaving the job of selecting fonts and setting margins to the machine.

The process of writing is as much one of editing as it is one of creation. My second important decision was to have the computer automatically track my changes and version history. I used a system called Git that allows me to take snapshots of my works at different points in time and attach descriptive messages to them. I can then browse this history of my changes and see how a work has changed over time. This makes it much easier to roll back changes if needed and explore tangents when I feel like it. It also prevents a dozen different files with obscure names littering up my computer. A summarized copy of the history is attached.

As I learned about writing poetry and fiction in class, I also learned about how to best use these tools and create a workflow that let me focus on writing and not get distracted by any sort of manual micromanagement. This is a continuing experiment but I'm glad I had the chance to put my ideas into practice.

As a final note, all my work in this class is being released under the terms of a Creative Commons Attribution Non-commercial Sharealike license. This license allows anyone to redistribute my work as long as it is non-commercial. It also allows others to creative derivative works based on my own as long as they give due credit. Though I may never publish on paper any of the works, I hope that someone someday might find them useful and maybe even interesting.

\footnotetext{
  This work is licensed under the Creative Commons Attribution-Noncommercial-Share Alike 3.0 Unported License. To view a copy of this license, visit \texttt{http://creativecommons.org/licenses/by-nc-sa/3.0/} or send a letter to Creative Commons, 171 Second Street, Suite 300, San Francisco, California, 94105, USA.}
\end{document}