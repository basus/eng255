\documentclass[12pt,letterpaper]{article}
\usepackage{setspace, fullpage, pslatex}

\title{Prologue}
\author{Shrutarshi Basu}
\date{\today}

\begin{document}

\maketitle
\doublespacing


I was a very shy kid in elementary school. Over the years, that has changed, thanks in no small part to some wonderful teachers who encouraged and helped me to push against my limits. As I've pushed the limits I've learned to always be on the lookout for new limits to push, new frontiers to explore. Writing fiction and poetry has been the most recent such frontier.

As a writer of technical non-fiction I've spent a number of years developing an eye for facts and details. I've been able to assume that my audience is mostly already interested in what I have to say. They are reading what I have written to get the facts of the matter (interspersed with my commentary) so that they can move on with their lives and use the information as they see fit. 

The move to fiction was interesting, to say the least. Crafting a piece that would gather and hold the reader's attention was a challenge. Using writing as a medium for emotional as well as informational transport was a skill that took considerable time and energy to learn and I am far from mastering it. While the class was focused on writing fiction, it was also a crash course in reading and understanding fiction. Having never taken a formal poetry or fiction class it was hard for me to appreciate quality and skill, especially in my own work. This realization was both uplifting and demoralizing. I found out that I actually wrote better than I thought I could. But I also realized that if I really wanted to continuing producing quality pieces, I would have much to read and learn.

Over the three months that I've spent training as a fiction writer, I've learned a lot of lessons many of which are not related to writing per se. Firstly, I rediscovered my love of writing. While my non-fiction writing was certainly interesting and fulfilling, writing fiction was a sideways step into a slightly thought process. It was a refreshing change which I certainly enjoyed.

Secondly, I came to terms with the fact that writing is hard, in more ways than one. Putting thoughts to paper can be difficult. Often there are no thoughts, or there are too many, or there are enough but none of them seem to belong on paper. And even when you have something to put down, you have to actually sit down and get it in ink (or in electrons as the case may be). That is much harder than it sounds. As a student college life interfered with my writing schedule far more than I would have liked. Whether it was things like two exams and a homework set in three days or more mundane things like running out of socks, there was always something else that needed doing. While I can't say that I have a solution, I did realize that a lot of times it's a matter of just sitting down and getting it over with. It's much easier to keep going than it is to get started and once you get ``into the Zone'' things can go surprisingly quickly.

Much of my other writing is done in isolation. I publish online and my readers sometimes comment or drop tips, but there is little real interaction and little of it is about writing. Being in a class full of other aspiring writers and being guided by an accomplished writer is a very different experience. It was certainly informative to have so many eyes looking at our work and getting useful, focused feedback. I am certainly a better writer in general and my specific pieces are much improved. But I also learned that you can't please everyone. I've decided to take Stephen King's advice of having one ideal reader and writing for that person. Though I don't have a single ideal reader yet, I have an ideal mindset that I find myself writing for.

My final portfolio is hopefully a reflection of the things I've learned. The poems reflect my experience learning to be both a reader and writer of poetry. I have attempted to move away from the stylized, formal style that I have had experience with to something more flexible and modern. My stories are the result of looking for my ``ideal reader''. I've found that whenever I think of fiction, it's almost like a movie playing inside my head and I've tried to reflect this direct and visual style in my prose.

At the end of the semester, I can safely say that though I've learned a lot and enjoyed the class greatly, I am not without regrets. I wish I had written more and read more and I wish I had asked more questions. But I'm ready to call this a frontier explored. I'm now looking forward to the next one: writing a longer, better fiction by the time I graduate and finding my way to the frontier after that one.

As a final note, all my work in this class is being released under the terms of a Creative Commons Attribution Non-commercial Sharealike license. This license allows anyone to redistribute my work as long as it is non-commercial. It also allows others to creative derivative works based on my own as long as they give due credit. Though I may never publish on paper any of the works, I hope that someone someday might find them useful and maybe even interesting.

\footnotetext{
  This work is licensed under the Creative Commons Attribution-Noncommercial-Share Alike 3.0 Unported License. To view a copy of this license, visit \texttt{http://creativecommons.org/licenses/by-nc-sa/3.0/} or send a letter to Creative Commons, 171 Second Street, Suite 300, San Francisco, California, 94105, USA.}
\end{document}