\documentclass[11pt,letterpaper]{article}
\usepackage{setspace, fullpage, pslatex}

\title{Notes for 9/11}
\author{Shrutarshi Basu}
\date{\today}

\begin{document}
\maketitle
\doublespacing


\section*{Existentialism}
Existentialism refers to the work of nineteenth and twentieth century philosophers who considered the human subject -- the thinking, feeling, acting, living human individual and their conditions of existence as the starting point of their philosophical thoughts. Existential philosophy is the ``explicit conceptual manifestation of an existential attitude''. Existentialists regard traditional systematic philosophy as too abstract and remote from human experience. Notable early existentialists were Soren Kierkegaard and Friedrich Nietzsche.

\section*{Analysis}
The first line is ``The first person is an existentialist'' -- this makes the following lines more meaningful as existentialism has at it's core the idea that the objective truths are too far remote from the actual human condition which is one of suffering and pain in an irrational world. The following lines make references to seemingly absurd situations or to conditions that are full of apathy, desolation and inherent, suppressed pain. ``seeing things the same'' might be a veiled reference to existentialist idea of using diversions to avoid boredom, though that doesn't make the comparison between Death Valley and the desert of Paran any clearer.

The starting line of the second part of the poem -- ``An earthquake a turret with arms and legs'' seems to be floating all by itself, it's not clear whether it's part of the first or second person's world view.

The lines associated with the beloved are interesting, because they take on an almost caring, fond tone of voice if seen in a particularly light. Taking the hit can be seen as the action of a beloved hero. The comparisons of Utah to Pakistan and Saudi Arabia can be seen as an attempt to equate the unknown with something familiar to understand it better. The planes of Miramar also have a heroic tone to them. However the remaining lines of the poem do not easily carry on the connection to the beloved.

The final reference to a third person as a Materialist again seems like an isolated line with no connection to the rest of the poem.

\section*{Personal thoughts}
I feel that this poem both the best and worst properties of poetry. The references to philosophies and emotions help to bring out ideas in a powerful, provocative way. But these connections can become very strained at times. The poem also seems to show a startling alck of connection. On the outset it seems like there is a clear division along the lines of the three persons. But not everything is clear cut inside these divisions. The first few lines are well connected by the theme of the particular person, but after that the connection becomes less obvious and requires some very deep thought to make sense of things. There are also the lines which seem completely dettached from everything around them and make almost no sense on their own.

The poem is a combination of powerful imagery and connections thrown in with lots of far more cryptic ideas and images. It almost seems at the poem is written at two levels: one to be understood with little or no effort and another to be not understood at all, except perhaps with great effort and concentration. The poem is almost a challenge to the reader, and a challenge that is not really meant to be overcome. While I don't grudge the poem its attempt to be a test of the intellect, I do feel that is somewhat contradictory to the purpose of a poem in general. 


\footnotetext{
  This work is licensed under the Creative Commons Attribution-Noncommercial-Share Alike 3.0 Unported License. To view a copy of this license, visit \texttt{http://creativecommons.org/licenses/by-nc-sa/3.0/} or send a letter to Creative Commons, 171 Second Street, Suite 300, San Francisco, California, 94105, USA.}
\end{document}